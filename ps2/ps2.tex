\documentclass[10pt,a4paper]{article}
\usepackage[utf8]{inputenc}
\usepackage{amsmath}
\usepackage{amsfonts}
\usepackage{amssymb}
\title{H.W. 1}
\author{Patrick Chen}

\begin{document}
\maketitle
\tableofcontents
\section{1}
a) 
The height of the recursion tree is n at LoD n, because SnowFlake-Edge decrements n once for each recursive call it makes. \\
b) The number of nodes corresponds to the number of edges at a given level. At each level of recursion, each edge splits into 4 edges. Since we start with 3 edges, there are $3 \cdot 4^i$ nodes.\\
c) There's only 1 triangle at each node, so the asymptotic rendering time is $\Theta(1)$.\\
d)At each level of the recursion tree, the number of triangles is equal to the number of edges from the last step, so it's 
$\Theta(num triangles)=\Theta(num edges)=\Theta(4^i)$\\
e) Since rendering triangles of arbitrary size all takes the same amount of time, the total asymptotic cost is jut the total number of triangles, which is $\Theta(4^n)$.
f) The height of the recursion tree is still $n$.\\
g) The number of nodes is still $3 \cdot 4^i$.\\
h) The line segments aren't rendered til the end, so the asymptotic rendering time is $0$.
i) $\Theta(1)$, since there is 1 line rendered for each node. \\
j) Nodes aren't rendered til the last level, so $0$.
k) The number of line segments that need to be rendered is a constant times the nnumber of nodes. The number of nodes is $3 \times 4^n$, so the asymptotic rendering time is $\Theta(4^n)$. \\
l) It's the cost of rendering the last level, which is $\Theta(4^n)$.\\
m) The height of the recursion tree is still $n$.\\
n) The number of nodes at a level is still $r \times 4^i$. \\
o) It's $0$ because nodes are only rendered at the last level.
p) Rasterizing a line segment takes resources proportional to the length of the line. At each level, the line segment lengths get divided by 3. Thus, it is $\theta((1/3)^n)$.\\
q)  It's $0$ since the lines aren't rasterized til the last level.\\
r)We have to multiply the number of nodes by the length of the segments, which results is $3 \times 4^n \times (1/3)^n= \Theta((4/3)^n)$ \\
s) The total cost is $\Theta((4/3)^n)$, summing over the previous two steps.
t)It's $\Theta(1)$, since the snowflake has finite surface area.
u)The side length of triangle corresponding a node at a given level $n$ for a triangle is $\Theta((1/3)^n)$. Then the area is $\theta((1/3)^{2n})$. At each level, we have $\Theta(3 \cdot 4^n)$ nodes. then the cost of rendering a given level is $\Theta((4/9)^n)$. Summing this over $n$ levels gives $\Theta(1)$.

\end{document}